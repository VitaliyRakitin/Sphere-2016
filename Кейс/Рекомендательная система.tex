\documentclass[a4paper,14pt]{article}
%%%%%%%%%%%%%%%%%%%%%%%  ПАКЕТЫ  %%%%%%%%%%%%%%%%%%%%%%%%%%%%%%%%%%%%%%%%%%%%%%
\usepackage{cmap}                               % Чтобы в PDF работал человеческий поиск
\usepackage[X2,T2A]{fontenc}                    % T2A = русская кодировка. X2 = яти
\usepackage[utf8]{inputenc}                     % Ввод в универсальной кодировке
\usepackage{setspace,soulutf8}      		        % Чтобы можно было менять межстрочный и межбуквенный интервалы
\usepackage{amsmath,amsfonts,amssymb,amsthm}    % Символы для математики
\usepackage{mathrsfs}                           % Символы для математики
\usepackage{dsfont}                             % Шрифт для полей действительных, комплексных... чисел командой \mathds
\usepackage[subfigure]{tocloft}  % Многоточие в оглавлении
\usepackage{array,multicol,multirow,bigstrut}   % Чтобы можно было делать в таблице колонки фиксированной ширины, слитные ячейки, вставлять strut'ы.
\usepackage{indentfirst}                        % Абзацный отступ везде
\usepackage[british,russian]{babel}             % Русские переносы, тире, типографика, самодержавие, духовность!
\usepackage[perpage]{footmisc}                  % Сброс счётчика сносок на каждой странице
\usepackage[pdftex,unicode,bookmarks=true,bookmarksopen=true,colorlinks=true,linkcolor=blue,urlcolor=blue,citecolor=blue]{hyperref} % Синие ссылки в PDF
\usepackage{microtype}                          % Свешивающаяся пунктуация и подгонка белого пространства по правилу \pm 2 процента
\usepackage{textcomp}                           % Чтобы в формулах можно было русские буквы писать через \text{}
\usepackage[paper=a4paper,top=12.7mm, bottom=12.7mm,left=12.7mm,right=12.7mm,bindingoffset=6.6mm,includefoot]{geometry} % Достаточно экономные размеры листа и поля для нумерации страниц внизу (для колонтитулов в стиле лекций по матану, нужно includefoot заменить на includehead и не только
\usepackage{xcolor}                             % Чтобы можно было цветные объекты вставлять
\usepackage[pdftex]{graphicx}                   % Чтобы вставились изображения
\usepackage{float,longtable}                    % Поддержка плавающих таблиц и рисунков
\usepackage[margin=0pt,font=small,labelfont=bf,labelsep=period]{caption} % Подписи таблиц и рисунком мелкие, жирные, с принятым в русской типографике разделителем.
\usepackage{rotating}                           % Создание своих акцентов, поворот объекта.
\usepackage{datetime}                           % Отображение времени
%\usepackage{embedfile}                         % Чтобы код LaTeXа включился как приложение в PDF-файл
\usepackage{xspace}
\usepackage{wrapfig,enumitem}                   % Обтекаемые текстом рисунки
\usepackage{mathtools}                          % В тексте используется smashoperator, чтобы избежать некрасивых пробелов вокруг сумм и пределов с большим подстрочником
\usepackage{cancel}                             % Красивое <<вычёркивание>> сокращающихся выражений одноимённой командой
\usepackage{tikz,pgfplots}			% Рисование графиков непосредственно кодом
\usepackage{subfigure}
\usepackage{fancyhdr}	% Пакет для создания колонтитулов в стиле лекций по матану
\usepackage{accents}
\usepackage{pb-diagram}

%%%%%%%%%%%%%%%%%%%%%%%  ПАРАМЕТРЫ  %%%%%%%%%%%%%%%%%%%%%%%%%%%%%%%%%%%%%%%%%%%
\setstretch{1}                          % Межстрочный интервал
\flushbottom                            % Эта команда заставляет LaTeX чуть растягивать строки, чтобы получить идеально прямоугольную страницу
\righthyphenmin=2                       % Разрешение переноса двух и более символов
\pagestyle{plain}                       % Нумерация страниц снизу по центру.
\settimeformat{hhmmsstime}              % Формат времени с секундами
\widowpenalty=300                       % Небольшое наказание за вдовствующую строку (одна строка абзаца на этой странице, остальное --- на следующей)
\clubpenalty=3000                       % Приличное наказание за сиротствующую строку (омерзительно висящая одинокая строка в начале страницы)
\setlength{\parindent}{1.5em}           % Красная строка.
\setlength{\topsep}{0pt}                % Уничтожение верхнего отступа, если он где проявится
%%%%%%%%%%%%%%%%%%%%%%%%%%%%%%%%%%%%%%%%%%%%%%%%%%%%%%%%%%%%%%%%%%%%%%%%%%%%%%%

%%%% Техническая подготовка для определения следующих команд %%%%
\makeatletter
\newcommand*{\relrelbarsep}{.386ex}
\newcommand*{\relrelbar}{%
  \mathrel{%
    \mathpalette\@relrelbar\relrelbarsep
  }%
}
\newcommand*{\@relrelbar}[2]{%
  \raise#2\hbox to 0pt{$\m@th#1\relbar$\hss}%
  \lower#2\hbox{$\m@th#1\relbar$}%
}
\providecommand*{\rightrightarrowsfill@}{%
  \arrowfill@\relrelbar\relrelbar\rightrightarrows
}
\providecommand*{\leftleftarrowsfill@}{%
  \arrowfill@\leftleftarrows\relrelbar\relrelbar
}
\providecommand*{\xrightrightarrows}[2][]{%
  \ext@arrow 0359\rightrightarrowsfill@{#1}{#2}%
}
\providecommand*{\xleftleftarrows}[2][]{%
  \ext@arrow 3095\leftleftarrowsfill@{#1}{#2}%
}
\makeatother



%%%%%%%%%%%%%%%%%%%%% Команды-сокращения %%%%%%%%%%%%%%%%%%%%%%%%%%%%%%%%%%%%%%
\newcommand{\pau}{\hskip .75em plus.1em minus.08em\relax}	% пробел для выражений с кванторами, например «\forall\ \e>0\pau\exists\ \delta\colon», где \e определяется как
\newcommand{\ds}{\displaystyle}			% Быстрое переключение на выключный стиль формулы, где интегралы и суммы большие
\newcommand{\e}{\varepsilon}                % эпсилон
\newcommand{\p}{\partial}                   % частная производная
\renewcommand{\phi}{\varphi}                % Чтоб фи писалась в соответствии с русской традицией
\newcommand{\q}{\varnothing}			% Пустое множество в русской традиции
\newcommand{\N}{\mathds{N}}			% Натуральные числа
\newcommand{\Z}{\mathds{Z}}			% Целые числа
\newcommand{\Q}{\mathds{Q}}			% Рациональные числа
\newcommand{\R}{\mathds{R}}			% Действительные числа
\renewcommand{\C}{\mathds{C}}			% Комплексные числа
\newcommand{\T}{\mathbb{T}}			% Отмеченное разбиение (для интегральных сумм Римана или Римана"--~Стилтьеса
\newcommand{\Rim}{\mathcal R}			% Риман
\renewcommand{\le}{\leqslant}           % Правильное меньше или равно
\renewcommand{\leq}{\leqslant}           % Правильное меньше или равно
\renewcommand{\ge}{\geqslant}           % Правильное больше или равно
\renewcommand{\geq}{\geqslant}           % Правильное больше или равно
\newcommand{\dd}{\setminus}		% Ассоциируется с diagdown, но это именно вычитание множеств
\renewcommand{\iff}{\,\Leftrightarrow\,}	% Если и только если
\newcommand{\imp}{\hspace{1pt plus1pt}\Rightarrow\hspace{1pt plus1pt}}		% Следовательно
\def\hm#1{#1\nobreak\discretionary{}{\hbox{$#1$}}{}} % Команда для переноса на следующую строку символов бинарных операций
\renewcommand{\a}{\langle} 		% Открывающая треугольная скобка
\newcommand{\s}{\rangle}		% Закрывающая треугольная скобка
\newcommand{\ol}[1]{\overline{#1}}	% Надчёркивание аргумента по всей ширине

%%%% СТРЕЛКИ %%%%%%%%%%%%%%
\renewcommand{\to}{\rightarrow}                   % Правильная стрелка вправо («стремится») без возможности подписать
\newcommand{\To}[1]{\xrightarrow{#1}}		% Стрелка «стремится», ширина которой зависит от ширины агрумента-подписи (подпись ставится вверху)
\newcommand{\tend}[1]{\xrightarrow[#1]{}}	% Стрелка «стремится», ширина которой зависит от ширины агрумента-подписи (подпись ставится внизу)
\newcommand{\Tot}[1]{\xrightarrow{\text{#1}}}	% Стрелка «стремится», ширина которой зависит от ширины текстового агрумента-подписи (подпись ставится вверху)
\newcommand{\te}[1][]{\xrightarrow[n\to\infty]{#1}}	% Стремится при n стремящемся к бесконечности
\newcommand{\TE}{\xrightarrow[N\to\infty]{}}	% Стремится при N стремящемся к бесконечности
\newcommand{\neV}{\overline{\vphantom{<}\quad}\kern-.35em\searrow}	% Не возрастает в стиле лекций по анализу
\newcommand{\neU}{\underline{\vphantom{<}\quad}\kern-.35em\nearrow}	% Не убывает
\newcommand{\rsh}[2][P]{\xrightrightarrows[{#2\to\infty}]{#1}}		% Двойная растяжимая стрелка «равномерно сходится последовательность». Снизу стрелки подпись-обязательный-аргумент, к которому приписывается автоматически стремление к бесконечности. Необязательные аргумент изменяет подпись сверху с P на то, что напишите
\newcommand{\rsH}[2][X]{\xrightrightarrows[{#2}]{#1}} % Равномерная сходимость по параметру. Подпись снизу задаётся явно и обязательно; подпись сверху задаётся необязательно, по умолчанию X
\newcommand{\nsh}[2][n]{\xrightarrow[#1\to\infty]{\|\ \|_{#2}}}		% Растяжимая стрелка «сходится по норме». Вид нормы подписывается в обязательном аргументе (например l_2). Необязательный аргумент «что стремится к бесконечности» (стремление к бесконечности приписывается автоматически)


%%%%% Предел как оператор %%%%%%%%
\newcommand{\yo}[2]{\lim\limits_{#1\to#2}}	% предел{при этой букве}{стремящейся сюда}
\newcommand{\prb}[1]{\lim\limits_{#1}}		% предел{по такой базе}

%%%%%% O-символика %%%%%%%%%%%%
\newcommand{\oo}{\overline{\overline o}} 		% o-малое без подписей и скобок (две черты сверху ставятся автоматически). Переопределение этой команды изменит вид всех надстроек, указанных ниже
\newcommand{\ou}{\underline{\underline{O}}}		% Аналогично O-большое
\newcommand{\ouu}[2]{\underset{#2\ }\ou(#1)}		% O-большое{по отношению к чему}{при каком процессе}
\newcommand{\ooo}[2]{\underset{#2\ }\oo(#1)} % o-малое{по отношению к чему}{при каком процессе}
\newcommand{\ooob}[2]{\underset{#2}{\oo\big(#1\big)}}	% Аналогично, только скобки вокруг первого аргумента с плавающим размером
\newcommand{\ooog}[2]{\underset{#2}{\ou\big(#1\big)}}

%%%%%% Дифференцирование %%%%%%%%%%%%%%
\newcommand{\CP}[2]{\frac{\partial #1}{\partial #2}}	% Частная производная{чего}{по чему}
\newcommand{\cP}[2]{\frac{d #1}{d #2}}			% Полная/материальная производная
\newcommand{\Jacoby}[4]{\begin{pmatrix}
\CP {{#1}_{1}}{{#2}_1} 	& \CP{{#1}_{1}}{{#2}_2} & \dots & \CP{{#1}_{1}}{{#2}_{#3}} \\[1ex]
\CP {{#1}_{2}}{{#2}_1} 	& \CP{{#1}_{2}}{{#2}_2} & \dots & \CP{{#1}_{2}}{{#2}_{#3}} \\
\vdots 			& \vdots		& \ddots& \vdots \\
\CP {{#1}_{#4}}{{#2}_1} & \CP{{#1}_{#4}}{{#2}_2}& \dots & \CP{{#1}_{#4}}{{#2}_{#3}}	
\end{pmatrix}}% матрица Якоби{чего}{по чему}{размерность образа}{размерность аргумента}


\newcommand{\ttilde}[1]{\tilde{\tilde{#1}}}	% Набрать символ(ы)-аргумент с двумя узкими волнами
\newcommand{\Til}[1]{\widetilde{#1}{} }		% Набрать символ(ы)-аргумент с волной, ширина которой равна ширине аргумента

%%%% ПОСЛЕДОВАТЕЛЬНОСТИ, СУММЫ, ПРОИЗВЕДЕНИЯ %%%%%
\newcommand{\pos}[2][n]{\big\{{#2}_{#1}\big\}_{#1=1}^\infty} 		% Последовательность: в фигурных скобках аргумент с нижним индексом по умолчанию n; справа от скобок подпись n (или необязательный аргумент) от 1 до бесконечности
\newcommand{\ar}[4]{\big\{{#1}_{#2}\big\}_{#2=#3}^{#4}} % Задание последовательности четырьмя аргументами: что, каким символом нумеруется, откуда, докуда
\newcommand{\Har}[4]{\left\{{#1}_{#2}\right\}_{#2=#3}^{#4}} % Аналог предыдущего, но размер фигурных скобок подстраивается под внутренность.
\newcommand{\AR}[3]{\big\{#1\big\}_{#2}^{#3}} 		% Последовательность. Три аргумента: {что и как нумеруеся}{номер=начальное значение}{конечное значение номера}
\newcommand{\ry}[3][1]{\sum\limits_{{#3}={#1}}^\infty {#2}_{#3}}	% Ряд[с какого номерая начиная]{чего}{как нумеруется}; по умолчанию с единицы; Например, \ry an
\newcommand{\rY}[2][1]{\sum\limits_{{#2}={#1}}^\infty} % Сумма до бесконечности[с какого номера начать]{символ индекса}; например, \rY n или \rY[0]j
\newcommand{\RY}[3]{\sum\limits_{#1 = #2}^{#3}}		% Сумма{по этому индексу}{от этого значения индекса}{до этого значения индекса}; Например, \RY n1N
\newcommand{\tmy}[3][1]{\prod\limits_{#3=#1}^{\infty}#2_{#3}}	% Бесконечное произведение (от слова times), аналог \ry
\newcommand{\tmY}[2][1]{\prod\limits_{#2=#1}^{\infty}} % аналог \rY
\newcommand{\TMY}[3]{\prod\limits_{#1=#2}^{#3}}		% аналог \RY



%%%%% Окружения для нумерованных перечней
\newenvironment{iItems}{\begin{enumerate}\let\AEtheenumi\theenumi{}\renewcommand{\theenumi}{\roman{enumi}}\renewcommand{\labelenumi}{(\theenumi)}}{\renewcommand{\labelenumi}{\theenumi.}\renewcommand{\theenumi}{\AEtheenumi}\end{enumerate}}	% (i), (ii), ...
\newenvironment{azItems}{\begin{enumerate}\let\AEtheenumi\theenumi{}\renewcommand{\theenumi}{\asbuk{enumi}}\renewcommand{\labelenumi}{(\theenumi)}}{\renewcommand{\labelenumi}{\theenumi.}\renewcommand{\theenumi}{\AEtheenumi}\end{enumerate}}	% (а), (б), ...
\newenvironment{roItems}{\begin{enumerate}\renewcommand{\labelenumi}{(\theenumi)}}{\renewcommand{\labelenumi}{\theenumi.}\end{enumerate}}	% (1), (2), ...
\newenvironment{oItems}{\begin{roItems}\setcounter{enumi}{-1}}{\end{roItems}}	% (0), (1), ...
%%% В КОНЦЕ ФАЙЛА ПРИВЕДЕНЫ КОМАНДЫ, КОТОРЫЕ Я НЕ УСПЕЛ ОПИСАТЬ

%%%%%%%%%%%%%%%%%%%% Операторы в смысле LaTeX %%%%%%%%%%%%%%%%%%%%%%%%%%%%%%%%%
\DeclareMathOperator{\sign}{sgn}	% Знак (перестановки)
\DeclareMathOperator{\sgn}{sgn}		% Сокращённая альтернатива sign
\DeclareMathOperator{\diag}{diag}	% Диагональная матрица
\DeclareMathOperator{\const}{const}	% Константа
\DeclareMathOperator{\rang}{rank}         % Оператор ранга
\DeclareMathOperator{\rank}{rank}	% На любой вкус
\DeclareMathOperator{\diam}{diam}	% Диаметр, например, разбиения
\DeclareMathOperator{\card}{card}	% Мощность множества
\DeclareMathOperator{\osc}{osc}		% Осцилляция	
\DeclareMathOperator{\grad}{grad}	% Градиент
\DeclareMathOperator{\intS}{int}	% Внутренность множества
\DeclareMathOperator{\ext}{ext}		% Внешность множества
\DeclareMathOperator{\supp}{supp}


%%%%%%% Математическая экзотика %%%%%
\newcommand{\defequiv}{\mathbin{\vbox{\baselineskip=1.95pt\lineskiplimit=0pt\hbox{.}\hbox{.}\hbox{.}}\hskip-0.3em\equiv}} 		% вертикальное троеточие =, то есть «по определению тождественно»
\newcommand{\rus}[1]{\mbox{\rm{\scriptsize{#1}}}}	% команда убивает зависимость русского текста в формуле от стиля абзаца (но и от стиля формулы тоже). Специально для обозначения левой и правой производных, чтобы  f'_п печаталось с прямой буквой «п» и нельзя было спутать с n
\newcommand{\overcirc}[1]{\accentset{\circ}{#1}}	% круг над символом
\newcommand{\overstar}[1]{\accentset{\bigstar}{#1}}	% звезда над символом

%%%%%%%%%%%%%%%%%%% Глобальные текстовые команды %%%%%%%%%%%%%%%%%%%%%%%%%%%%%%
\newcommand{\ENGs}[1]{\foreignlanguage{british}{#1}} % Переключение правил переноса и прочих правил автооформления на английский язык действует на аргумент
\newcommand{\ENG}{\selectlanguage{british}} % Глобальное переключение (если просто начать писать на другом язык, велика вероятность появления очень больших пробелов или помещения части слова за границу листа
\newcommand{\RUS}{\selectlanguage{russian}}
\newcommand{\fnnsp}{\hspace{-0.4em}} % Знак сноски принято ставить до всех знаков препинания, кроме ! ? ...
% А если есть возможность подвинуть низкий знак препинания под сноску, то для этого Андрей Викторович и придумал \fnnsp
\let\myfootnote\footnote
\renewcommand{\footnote}[1]{\myfootnote{\;#1}} % Сноски Андрея Виторовича

%%%%%%%% Определение разрядки разреженного текста и задание красивых и притом регулируемых многоточий
% Зачем и почему описано в блоге http://kostyrka.ru/blog
\newdimen\ellipsiskern
\setlength{\ellipsiskern}{.1em}
\newdimen\ellipsiskernen
\setlength{\ellipsiskernen}{.2em}
\newcommand{\ldotst}{.\kern\ellipsiskern.\kern\ellipsiskern.}	% вместо ... пишем \ldotst{}
\newcommand{\ldotse}{!\kern\ellipsiskern.\kern\ellipsiskern.}	% вместо !.. пишем \ldotse{}
\newcommand{\ldotsq}{?\kern\ellipsiskern\kern-.11em.\kern\ellipsiskern.}	% вместо ?.. пишем \ldotsq{}
\newcommand{\ldotsten}{.\kern\ellipsiskernen.\kern\ellipsiskernen.}		% аглийское ...
\newcommand{\ldotspen}{.\kern\ellipsiskernen.\kern\ellipsiskernen.\kern\ellipsiskernen\kern.15em.}
\newcommand{\ldotseen}{.\kern\ellipsiskernen.\kern\ellipsiskernen.\kern\ellipsiskernen\kern.15em!}
\newcommand{\ldotsqen}{.\kern\ellipsiskernen.\kern\ellipsiskernen.\kern\ellipsiskernen\kern.067em?}
%%%%%%%%%%%%%%%%%%%%%%%%%%%%%%%%%%%%%%%%%%%%%%%%%%%%%%%%%%%%%%%%%%%%%%%%%%%%%%%



%%%%%%%%%%%%%%%%%%%%%%% ДРУГИЕ ПОЛЕЗНЫЕ КОМАНДЫ БЕЗ КОММЕНТАРИЕВ (ПОКА) %%%%%%%
\newcounter{rad}
\setcounter{rad}{1}
% a_1+a_2+\ldots
	\def\rad#1#2{\ifnum \therad < \the\numexpr(#2 + 1)\relax #1_\therad + \addtocounter{rad}{1} \rad{#1}{#2}\else \ldots \fi \setcounter{rad}{1}}
%a_{N+1}+a_{N+2}+\ldots
	\def\raD#1#2#3{\ifnum \therad < \the\numexpr(#2 + 1)\relax #1_{#3{\therad}} + \addtocounter{rad}{1} \raD{#1}{#2}{#3}\else \ldots \fi \setcounter{rad}{1}}
\def\crad#1#2#3#4{\ifnum \therad < \the\numexpr(#4 + 1)\relax #1{\therad}#3 + \addtocounter{rad}{1} \crad{#1}{#2}{#3}{#4}\else \ldots +#1{#2}#3 \fi \setcounter{rad}{1}}
\def\craD#1#2#3#4#5#6{\ifnum \therad < \the\numexpr(#4 + 1)\relax #1{#5{\therad}#6}#3 + \addtocounter{rad}{1} \craD{#1}{#2}{#3}{#4}{#5}{#6}\else \ldots +#1{#5#2#6}#3  \fi \setcounter{rad}{1}}
%repeat argument #1 #2 times
	\def\tfT#1#2{\ifnum \therad < \the\numexpr(#2 + 1)\relax #1\addtocounter{rad}{1} \tfT{#1}{#2}\else \fi \setcounter{rad}{1}}
\def\dI#1#2#3#4#5#6{\ifnum \therad < \the\numexpr(#6 + 1)\relax {#1}_{\therad}{#3}_{\therad}#5\addtocounter{rad}{1} \dI{#1}{#2}{#3}{#4}{#5}{#6}\else \dots #5{#1}_{#2}{#3}_{#4}\fi \setcounter{rad}{1}}
\def\DI#1#2#3#4#5#6{\ifnum \therad < \the\numexpr(#6 + 1)\relax #1\therad#3\therad#5\addtocounter{rad}{1} \DI{#1}{#2}{#3}{#4}{#5}{#6}\else \dots #5#1#2#3#4\fi \setcounter{rad}{1}}
\def\iSum#1#2#3{\ifnum \therad < #3\relax {#1}_{\therad}{#2}_{\the\numexpr(#3 - \therad + 1)\relax}+ \addtocounter{rad}{1}\iSum{#1}{#2}{#3}\else {#1}_{\therad}{#2}_{1}\fi \setcounter{rad}{1}}
\def\tms#1#2{\ifnum \therad < \the\numexpr(#2 + 1)\relax #1_\therad \ifnum \therad < \the\numexpr(#2)\relax \cdot \else \fi\addtocounter{rad}{1} \tms{#1}{#2}\else \cdots \fi \setcounter{rad}{1}}
\def\ctms#1#2#3#4#5{\ifnum \therad < \the\numexpr(#4 + 1)\relax #1{#2\therad}#3 \ifnum \therad < \the\numexpr(#4)\relax \cdot \else \fi\addtocounter{rad}{1} \ctms{#1}{#2}{#3}{#4}{#5}\else \cdots #1{#2#5}#3\fi \setcounter{rad}{1}}
\newcommand{\cSize}{\footnotesize}
\newcommand{\cmt}[1]{\text{\cSize #1}}
\newcommand{\BIggl}[1]{\left#1\vphantom{\Bigg(\frac12\Bigg)^2}\right.}
\newcommand{\BIggr}[1]{\left.\vphantom{\Bigg(\frac12\Bigg)^2}\right#1}

\newcommand{\NN}{\mathcal{N}} %нормальное распределенеие

\newtheorem{The}{Теорема}[section]
\newtheorem{Ut}{Утверждение}[section]
\newtheorem{Sl}{Следствие}[section]
\newtheorem{Task}{Упражнение}[section]
\newtheorem{Def}{Определение}[section]
\newtheorem{Lem}{Лемма}[section]
\newtheorem{Pre}{Предложение}[section]
\newtheorem{Zam}{Замечание}[section]
\newtheorem{Zad}{Задача}[section]
\newenvironment{Proof}
       {\par\noindent{\textbf{Доказательство.}}}
       {\hfill$\scriptstyle\blacksquare$}
\newenvironment{Solution}
       {\par\noindent{\textbf{Решение.}}}
       {\hfill$\scriptstyle\blacksquare$}

\newtheorem{Examp}{Пример}[section]


\usepackage[procnames]{listings}
\usepackage{color}
 
\definecolor{keywords}{RGB}{255,0,90}
\definecolor{comments}{RGB}{0,0,113}
\definecolor{red}{RGB}{160,0,0}
\definecolor{green}{RGB}{0,150,0}
 
\lstset{language=Python, 
        basicstyle=\ttfamily\small, 
        keywordstyle=\color{keywords},
        commentstyle=\color{comments},
        stringstyle=\color{red},
        showstringspaces=false,
        identifierstyle=\color{green},
        procnamekeys={def,class}}
 

\usepackage{caption}
\usepackage{pdfpages}

\begin{document}
\begin{center}
{\LARGE \bf Рекомендательная система \\}
{\Large \copyright Ракитин Виталий Павлович, \\}
{\Large МГУ им. М.В.Ломоносова, 5 курс, механико-математический факультет.\\
15 сентября 2016\\}
\end{center}
\tableofcontents
\section{Цель }
Проектировка рекомендательной системы.

\section{Постановка задачи }
{\it \bf Что мы имеем?}
\begin{enumerate}
\item Множество пользователей $u = u(\Psi) \in U(\Psi)= \{u_1(\Psi),u_2(\Psi),\dots\}$, где $\Psi$ "--- множество параметров, характеризующих данного пользователя:
\begin{itemize}
\item Пол;
\item Возраст;
\item Социальный статус, социальная группа;
\item Регион, где он проживает;
\item Увлечения, интересы;
\item \dots
\end{itemize}
\item Множество объектов  $obj \in \Omega = \{obj_1,obj_2,\dots\}$;
\item Множество действий над объектами $ r_{ij} = (i = u, j = obj) \in \Lambda = \{r_{11}, r_{12}, \dots\}$\\ (оценка, покупка, просмотр и тд "--- {\bf рейтинг}).
\end{enumerate} 
\begin{Zam} 
Из множества действий можно так же генерировать элементы множества $\Psi$\\ (пользователи оптимисты, пользователи со схожими препочтениями \dots)
\end{Zam}

{\it \bf Что мы хотим получить?}

Предсказать предпочтения конкретного пользователя $u \in U$, а так же составить для него {\bf персональные рекомендации} на основе этих предпочтений.

%\section{Решение задачи}
%Составим матрицу <<предпочтений>> следующим образом:
%\begin{itemize}
%\item Строки "--- пользователи $U$;
%\item Столбцы "--- объекты $\Omega$;
%\item На пересечении "--- результат действия $\Lambda$.
%\end{itemize}
%\begin{center}
%\begin{tabular}[t]{|c|c|c|c|c|c|}
%\hline
%    & $obj_1$ & $obj_2$ & $obj_3$ & $obj_4$ & $obj_5$  \\
%\hline  
%$u_1$ & 4 & 5  & 3 & ? &  2  \\  
%hline 
%$u_2$ & 2  & ? & 1 & 4 & 2  \\ 
%\hline
%$u_3$ & 3  & 7 & 5 & 1 & 3  \\ 
% \hline
%\end{tabular}
%\end{center}

%В тех позициях, где у нас есть конкретное значение действия, подразумевается, что пользователь уже~проявил <<явную>> активность, а пропущенные значения нам необходимо <<предсказать>>, то есть найти $\ol r_{ij}$
\section{Сбор информации о пользователе }
Чем больше информации мы будем иметь "--- тем больше вероятность получить точное предсказание.
\begin{itemize}\label{inform}
\item При регистраиции предложить заполнить анекту с минимальной информацией о пользователе;
\item Предложить синхронизацию социальных сетей;
\item Отслеживать параметры <<друзей>>, а так же их предпочтения;
\item Следить за <<явной>> активностью пользователя (покупки, лайки);
\item Фиксировать <<неявную>> активность (какие страницы посещает, на каких задерживается).
\end{itemize}

\section{Введение связей между объектами} 
{\bf Идея:} Нет смысла предлагать пользователям товар, который он уже преобрёл (оценил) или строго аналогичный, однако можно рекомендовать <<родственные>> объекты, например
\begin{itemize}
\item {\it Если {\bf u} только преобрёл iPhone 6s, то нет смысла предлагать ему телефоны, но можно предложить аксессуары};
\item {\it Бессмысленно предлагать читать пользователю одну и ту же запись, опубликованную в разных местах, но можно предложить записи схожей тематики.}
\item \dots
\end{itemize}

Для каждого объекта введём 2 параметра:
\begin{enumerate}
\item {\bf Тип;}
\item {\bf Родство}.
\end{enumerate}

Одинаковые (аналогичные) объекты нумеруются одинаковым {\bf типом}, <<близкие>> "--- {\bf родством}. Родство может задаваться как {\bf <<базовое>>} (аксессуары, товары одного производителя/автора, с помощью хэш-тегов и тд), а так же его можно определять из соображения, какие объекты обычно оцениваются (преобретаются) вместе. После этого проведём кластеризацию всех объектов по степени родства.

\section{Способы хранения данных}
\subsection{Объекты}\label{obj}
\begin{center}
\begin{tabular}[t]{|c|c|c|c|c|c|c|c|c|c|}
\hline
 Объект($\Omega$) & origID & Keys & TopObj & Тип &\multicolumn{3}{|c|}{Родство} & Time & Click \\\cline{6-8}
   &  && && subID & similarity &  TopObj & & \\
\hline  
 &  &  & &&  &  & &&\\ 
\end{tabular}
\end{center}
\begin{itemize}
\item {\bf origID} "--- кодовый номер объекта;
\item {\bf Keys }"--- ключевые слова, хэш-теги, параметры, по которым определяется степень родства с другими объектами;
\item {\bf Click }"--- количество обращений;
\item {\bf Time }"--- время последнего обращения к объекту;
\item {\bf TopObj} "--- популярность данного объекта:
\[TopObj = function(Time,Click)\]
\item {\bf Родство} "--- связь c другими объектами;
\item {\bf Similarity} "--- степень родства объектов;
\item {\bf subID} "--- кодовый номер родственного объекта \\(отсортированы по степени родства, в случае одинаковой --- по Top);
\end{itemize}

\subsection{Top-популярных объектов}\label{pop}
Отдельно храним список объектов, отсортированный по популярности, на случай абсолютно холодного старта.
\begin{center}
\begin{tabular}[t]{|c|c|c|c|c|c|c|}
\hline
 origID & Click \\
\hline  
 &   \\ 
\end{tabular}
\end{center}
\begin{itemize}
\item {\bf origID} "--- кодовый номер объекта;
\item {\bf Click} "--- количество обращений;
\end{itemize}

\subsection{Пользователи} \label{users}
\begin{center}
\begin{tabular}[t]{|c|c|c|c|c|c|c|}
\hline
 User (U) & uID & $\mathrm{origID_i}$ & $\mathrm{r_{ui}}$ & \dots Параметры($\Psi$) \dots & ObjTopID \\
\hline  
 &  &  & & &  \\ 
\end{tabular}
\end{center}
\begin{itemize}
\item {\bf uID} "--- уникальный номер пользователя;
\item {\bf $\mathrm{origID_i}$} "--- все объекты, с которыми пользователь взаимодействовал; 
\item {\bf $\mathrm{r_{ui}}$} "--- рейтинг данных объектов;
\item {\bf Параметры} "--- все параметры которые мы смогли узнать о нашем пользователе;
\item {\bf ObjTopID} "--- Top-10 наиболее релевантных рекомендаций для данного пользователя;
\end{itemize}

\subsection{База сходства пользователей}
\begin{center}
\begin{tabular}[t]{|c|c|c|c|c|c|}
\hline
    & $u_1$ & $u_2$ & $u_3$ & $u_4$ & \dots  \\
\hline  
$u_1$ &  &  &  &  &   \\  
 \hline
\dots &  &  &  &  &   \\ 
 \hline
\end{tabular}
\end{center}
\begin{itemize}
\item По строкам и столбцам распределены все пользователи конкретного кластера;
\item На пересечении "--- степени их похожести $sim(u_i,u_j) = \frac{1}{1+dH\left(u_i,u_j\right)}$ (через растояние Хэмминга).
\end{itemize}
\subsection{База предпочтений}
Для каждого отдельно взятого пользовательского кластера строим таблицу следующего вида:
\begin{center}
\begin{tabular}[t]{|c|c|c|c|c|c|}
\hline
    & $obj_1$ & $obj_2$ & $obj_3$ & $obj_4$ & \dots  \\
\hline  
$middle$ & 4 &  & 5 & 6 &   \\  
\hline  
$u_1$ & 3 &  & 7 & 9 &   \\  
\hline 
$u_2$ & 6 & 1  & 8 & 3 &    \\ 
 \hline
$u_3$ &  &  &  2&  &   \\ 
 \hline
\dots &  &  &  &  &   \\ 
 \hline
\end{tabular}
\end{center}
\begin{itemize}
\item По строкам распределены все пользователи данного кластера;
\item По столбцам "--- список всех объектов, с которыми взаимодействуют данные пользователи, а так же по N первых родственных объектов другого типа по степени родства.
\item На пересечении расположены {\bf рейтинги}, полученные от соответствующих пользователей соответствующими объектами($\Lambda$). На местах пропуска впишем {\bf оценку рейтинга} (наше <<предсказание>>).  
\item В строке {\bf middle} указаны средние оценки объектов по всем пользователям кластера.
\end{itemize}

\subsection{База кластеров пользователей}
\begin{center}
\begin{tabular}[t]{|c|c|c|c|c|c|c|}
\hline
  Номер кластера  & uIDs & Центроид  \\
\hline  
& & \\  

\end{tabular}
\end{center}
\begin{itemize}
\item uIDs "--- списки польователей в данном кластере;
\end{itemize}

\section{Роботы}
\begin{Def}
Роботы "--- программы-демоны, обеспечивающие работу системы.
\end{Def}
\begin{enumerate}
\item {\bf Кластеризатор пользователей} "--- проходит по базе (\ref{users}), случайным образом выбирая начальные центры кластеров. Сохраняем $J$ и $\Theta$ (ошибка и центроиды) для минимального $J$. Так как алгорит {\bf k-means} является локальным, то проводим данную операцию большое количество раз;
\item {\bf Редактор базы кластеров} "--- запускается, когда  {\bf кластеризатор Users} нашёл разбиение c меньшей ошибкой;
%\item{\bf Редактор базы пользователей  } "--- запускается, когда  {\bf редактор базы кластеров} изменил список кластеров и их состав. Изменяет номер текущего кластера для каждого пользователя; 
\item{\bf Редактор базы пользователей сходства } "--- запускается, когда  {\bf редактор базы кластеров} изменил список кластеров и их состав;
\item {\bf Редактор базы объектов} "--- добавляет новые объекты в базу  (\ref{obj}), пересчитывает Родство объетов, определяет тип;
\item {\bf Информатор} "--- при выполнении пользователями действий над объектами увеличивает {\bf Click} и {\bf TopObj}, изменяет {\bf Time}, добавляет в базу (\ref{users}) информацию о взаимодействии объектов, повышает количество {\bf Click} в (\ref{pop});
\item {\bf Сортировщик Top} "--- регулярно сортирует базу популярных объектов;
\item {\bf Редактор базы рекомендаций} "--- запускаетя при изменении базы сходства пользователей или базы объектов,  высчитывает Top-10 рекомендаций для каждого пользователя;
\item{\bf Редактор базы пользователей  } "--- получает от {\bf редактора базы рекомендаций} Top-10 и заносит его~в~базу (\ref{users});
\item {\bf Мусорщик} "--- следит за {\bf Time} в списке объектов и удаляет устаревшие, а так же все упоминания о них в других базах. 
\end{enumerate}

\section{Построение рекомендаций}
\subsection{Кластеризация пользователей } 
{\bf Идея:} похожим пользователям обычно нравятся похожие объекты, поэтому на основе пункта (\ref{inform}) кластеризуем множество $U$.

\begin{itemize}
\item Метрика "--- расстояние Хэмминга, а именно $dH(x,y)$ "--- количество различных компонент в {\bf x} и {\bf y};
\item Применим метод кластеризации {\bf k-means};
\item Количество кластеров будем определять исходя из количества пользоваталей;
\item Сохраним центроиды наших кластеров в множество $\Theta = \{c_1, c_2, \dots\}$.
\item Мера ошибки
\[
{\ol J} = \sum\limits_{n=1}^N \sum\limits_{k=1}^k r_{nk} d(x_n,\mu_k),
\]
где $d(x_n,\mu_k)$ "--- функция расстояния, 
$\mu_k$ "--- один из объектов кластера.
\end{itemize}
\subsection{Предсказание}
Рассмотрим 2 подхода построения предсказания. Оба запроса будут выполняться на разных серверах. В качестве результирующего будем брать тот, который приходит быстрее, либо среднее арифметическое результатов.
\subsubsection{Пользовательский}
\[
\ol r_{ui} = r_{u} + \frac{\sum\limits_{u\in U_i} sim(u,v)\cdot(r_{vi} - r_v)}{\sum\limits_{u\in U_i} sim(u,v)}
\]
\begin{itemize}
\item $r_u$ "--- средняя оценка нашим пользователем всех объектом;
\end{itemize}
\subsubsection{Объектный}
\[
\ol r_{ui} = r_{i} + \frac{\sum\limits_{j\in I_u} sim(i,j)\cdot(r_{ui} - r_j)}{\sum\limits_{j\in I_u} sim(i,j)}
\]
\begin{itemize}
\item $r_j$ "--- средняя оценка пользователями одного кластера данного объекта;
\end{itemize}

\subsection{Конечный алгоритм подбора рекомендаций}
Рассмотрим несколько вариантов восприятия пользователя в системе:
\begin{enumerate}
\item {\it Пользователь существует в базе, информация по нему собрана:}
\begin{itemize}
\item {\it Существует список Top-10 рекомендаций} "--- рекомендуем первые объекты из топа;
\item {\it Списка рекомендаций нет} "--- возьмём выборку пользователей из того же кластера из базы предпочтений, на их основе сделаем предсказания. Отсортируем предсказания по рейтингу и предложим Top;
\end{itemize} 
\item Пользователя нет в базе:
\begin{itemize}
\item {\it Можно собрать <<неявную>> информацию} "--- определим ближайшую группу из базы кластеров, возьмём выборку пользователей из этого кластера из базы предпочтений, на их основе сделаем предсказания. Отсортируем предсказания по рейтингу и предложим Top;
\item {\it Холодный старт } "--- предложить Top-популярных;
\end{itemize} 
\end{enumerate}

\end{document}
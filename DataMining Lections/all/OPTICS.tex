\subsection{OPTICS}
OPTICS "--- {\it <<ordering points to identify the clustering structure>>. }


{\bf Основная идея} "--- не ограничиваться фиксированной плотностью, как в DBSCAN, а варьировать ее в зависимости от того, как много объектов попадают в $\e$-окрестность.

Как в алгоритме DBSCAN, OPTICS требует двух параметров: $\e$, который описывает максимальное расстояние (радиус) для оценки плотности, и $MinPts$ "--- та самая минимальная плотность, то есть количество объектов, необходимых для формирования кластера. 
Однакое, в отличии от DBSCAN, OPTICS также учитывает то, что точки могут являться частью более плотного кластера, поэтому вводится понятие $core\_distance$, которое описывает расстояние до ближайшей точки внутри круга радиуса $MinPts$.
\[
core\_distance(p) = 
\begin{cases}
\mathrm{UNDEFINED},\quad \text{if } |N_{\e} (p)| < MinPts;\\
\text{{\it MinPts} is the smallest distance to } N_{\e}(p), \quad \mathrm{otherwise}.
\end{cases}
\]

Так же введём понятие $reachability\_distance$, описывающее расстояние от объекта $O$ до $p$ или $core\_distance(p)$:
\[
reachability\_distance(p) = 
\begin{cases}
\mathrm{UNDEFINED},\quad \text{if } |N_{\e} (p)| < MinPts;\\
\max(core\_dist(p),dist(p,O)), \quad \mathrm{otherwise}.
\end{cases}
\]

Если окажется так, что $O$ и $P$ "--- ближайшие друг к другу объекты, то необходимо, что бы в конечном итоге они оказались в одном кластере.

Оба <<новых>> параметра будут неопределены, если в районе данной точки не будет распологаться достаточно плотный кластер, удовлетворяющий основным параметрам ($\e, MinPts$).

\subsubsection{Алгоритм OPTICS}

Рассмотрим данный алгоритм в виде псевдокода на языке $Python$:
\begin{lstlisting}
function optics(X, eps, min_pts)
    for each point x in X:
        x.reachability-distance = UNDEFINED
    for each unprocessed point x in X:
        N = getNeighbors(x, eps)
        mark x as processed
        output x to the ordered list
        if (core-distance(x, eps, min_pts) != UNDEFINED):
            Seeds = empty priority queue
            update(N, x, Seeds, eps, min_pts)
            for each next z in Seeds:
                N1 = getNeighbors(z, eps)
                mark z as processed
                output z to the ordered list
                if (core-distance(z, eps, min_pts) != UNDEFINED):
                    update(N1, z, Seeds, eps, min_pts)

function update(N, x, Seeds, eps, min_pts)
    coredist = core-distance(x, eps, min_pts)
    for each z in N:
        if (z is not processed):
            new-reach-dist = max(coredist, dist(x, z))
            # z is not in Seeds
            if (z.reachability-distance == UNDEFINED):
                z.reachability-distance = new-reach-dist
                Seeds.insert(o, new-reach-dist)
            # z in Seeds, check for improvement
            else:
                if (new-reach-dist < z.reachability-distance):
                    z.reachability-distance = new-reach-dist
                    Seeds.move-up(z, new-reach-dist)
\end{lstlisting}